\chapitre{Op�rateurs diff�rentiels}
\titre{Op�rateur gradient}
\[\D{U} = \grad{U} \cdot \overrightarrow{\D{l}} \]
\[\grad{U} = \left(
\begin{array}{c}
	\partd{U}{x} \\[7pt]
	\partd{U}{y} \\[7pt]
	\partd{U}{z}
\end{array} \right) \hspace{2cm}
\grad{U} = \left(
\begin{array}{c}
	\partd{U}{r} \\[7pt]
	\frac{1}{r}\,\partd{U}{\theta} \\[7pt]
	\partd{U}{z}
\end{array} \right) \hspace{2cm}
\grad{U} = \left(
\begin{array}{c}
	\partd{U}{r} \\[7pt]
	\frac{1}{r}\partd{U}{\theta} \\[7pt]
	\frac{1}{r\,\sin\theta}\partd{U}{\varphi}
\end{array} \right)
\]
\[\grad{U} = \vec{\nabla}U\ \ \text{o� (en cart�siennes)} \ \vec{\nabla}=\left(
\begin{array}{c}
	\partd{}{x} \\[7pt]
	\partd{}{y} \\[7pt]
	\partd{}{z}
\end{array} \right)
\]
\[ \text{Formule du gradient: } \oiint_{(\Sigma)}U(M,t)\overrightarrow{\D{S}} = \iiint_{(V)}\grad{U(M,t)}\D{\tau} \]
\titre{Op�rateur divergence}
\[ \oiint_{(\Sigma)}\vec{a}(M,t)\cdot \overrightarrow{\D S} = \iiint_{(V)}\div{\vec{a}(M,t)}\D \tau \ ,\forall V \]
\[\div{\vec{a}}=\partd{a_x}{x}+\partd{a_y}{y}+\partd{a_z}{z}\]
\[\div{\vec{a}}= \vec{\nabla} \cdot \vec{a}\]
\titre{Op�rateur rotationnel}
\[\oint_{(\mathscr{C})}\vec{a}(M,t) \cdot \overrightarrow{\D{l}} = \iint_{(\mathscr{S})}\rot{\vec{a}(M,t)} \cdot \overrightarrow{\D{S}} \]
\[\rot{\vec{a}} = \vec{\nabla} \wedge \vec{a}\]
\titre{Op�rateur laplacien}
\[ \Delta U = \div{\left(\grad{U}\right)} \hspace{1cm} \vec{\Delta}\vec{a}=\grad{\left(\div{\vec{a}}\right)}-\rot{\left(\rot{\vec{a}}\right)}=\nabla^2\vec{a}\]
\[\text{En cart�siennes: }\Delta U=\nabla^2 U= \partd{^2U}{x^2}+\partd{^2U}{y^2}+\partd{^2U}{z^2} \hspace{1cm} \vec{\Delta}\vec{a}
\left(
\begin{array}{c}
	\Delta a_x \\
	\Delta a_y \\
	\Delta a_z
\end{array}
\right) \hspace{1cm}
\Delta\left( \frac{1}{r} \right) =0\]
\titre{Quelques formules de calcul}
\indent $\circledast$ � savoir par coeur:
\[\rot{\left( \grad{U} \right)} =0 \hspace{15mm} \div{\left( \rot{\vec{a}} \right) =0} \hspace{15mm} \rot{\left( \rot{\vec{a}} \right)}=\grad{\left( \div{\vec{a}} \right)}-\vec{\Delta}\vec{a}\]
\[\vec{u}\wedge(\vec{v}\wedge \vec{w})=(\vec{u}\cdot \vec{w})\vec{v}-(\vec{u}\cdot \vec{v})\vec{w}
\]
\indent $\circledast$ � savoir retrouver:
\[\grad{(U\;V)}=U\:\grad{V}+V\:\grad{U} \hspace{15mm} \div{(\vec{a}\wedge\vec{b})}=\vec{b}\cdot\rot{\vec{a}}-\vec{a}\cdot\rot{\vec{b}}
\]
\titre{D�riv�e particulaire}
\[ \frac{D\,\vec{a}}{D\,t} = \partd{\vec{a}}{t}+\left(\vec{a}\cdot\grad{}\right)\left(\vec{a}\right) = \partd{\vec{a}}{t}+\grad{\left(\frac{a^2}{2}\right)}-\vec{a}\wedge\rot{\vec{a}}
\]