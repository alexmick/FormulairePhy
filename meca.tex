\chapitre{M�canique}
\titre{Mouvement � forces centrales}
\titre{R�duction du probl�me � 2 corps}
\[
\vec{r}=r\,\vec{e}_r=\overrightarrow{M_1M_2}\hspace{10mm}
\mu=\frac{m_1\,m_2}{m_1+m_2}\ (masse\ r\acute{e}duite) \hspace{10mm}
\mu \ddot{\vec{v}}=f_G(r)\vec{e}_r \Rightarrow \vec{r}(t) \left\{
\begin{array}{c}
	\vec{r}(t)=\vec{r}_1(t)+\vec{r}_2(t) \\
	m_1\vec{r}_1(t)+m_2\vec{r}_2(t)
\end{array} \right.
\]
\titre{Cin�tique des syst�mes mat�riels}
\[
\vec{v}= \left(
\begin{array}{c}
	\dot{r} \\
	r\,\dot{\theta} \\
	\dot{z}
\end{array} \right) \hspace{20mm}
\vec{a} = \left(
\begin{array}{c}
	\ddot{r}-r\,\dot{\theta}^2 \\
	r\,\ddot{\theta}+2\dot{r}\,\dot{\theta} \\
	\ddot{z}
\end{array} \right) \hspace{20mm}
\vec{v}=\frac{\D s}{\D t}\;\vec{T} \hspace{10mm} \vec{a}=\vec{a}_T+\vec{a}_N=\frac{\D v}{\D t}\;\vec{T}+\frac{v^2}{\mathbb{R}}\vec{N}
\]
\[
\vec{\sigma}_A=\vec{\sigma}_B+\overrightarrow{AB}\wedge m\,\vec{v}_g\ (\textsc{Varignon}) \hspace{15mm}
\vec{\sigma}_A=\vec{\sigma}^{\star}+\overrightarrow{AG}\wedge m\,\vec{v}_g \hspace{15mm}
E_c=E_{c}^{\star}+\frac{1}{2}m\;v_G^2\ (\textsc{K�nig})
\]
\titre{Cin�matique du solide}
\[
\vec{v}_P=\vec{v}_Q+\overrightarrow{PQ}\wedge\vec{\Omega}_\mathcal{S} \hspace{15mm}
\sigma_\Delta=\mathcal{J}_\Delta\:\Omega \hspace{15mm}
E_c=\mathcal{J}_\Delta\frac{\Omega^2}{2} \hspace{15mm}
\overrightarrow{M_A}=\overrightarrow{M_B}+\overrightarrow{BA}\wedge \vec{F}
\]
\titre{Lois de \textsc{Coulomb}}
\[
\text{Glissement (oppos� au mouvement): } \left\|\vec{T}\right\|=f\,\left\|\vec{N}\right\| \hspace{10mm}
\text{Non glissement: } \left\|\vec{T}\right\|\leq f\,\left\|\vec{N}\right\|
\]
\titre{Puissance et travail}
\[
\mathscr{P}_{glisseur}=\vec{F}\cdot \vec{v}_{pt\ app} \hspace{15mm}
\mathscr{P}_{couple}=\vec{\Omega}\cdot \vec{\Gamma} \hspace{15mm}
\delta W_{conserv}=-\D E_px
\]
\[
\frac{\D E_c}{\D t}=\frac{\delta W_{int}}{\D t}+\frac{\delta W_{ext}}{\D t}=\mathscr{P}_{int}+\mathscr{P}_{ext}
\]