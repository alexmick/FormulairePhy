\chapitre{�l�ctromagn�tisme}
\titre{�lectrocin�tique}
\[i=C\diff{U_c}{t} \hspace{15mm} q=C\,U \hspace{15mm}U=L\diff{i}{t} \hspace{15mm} e=r\eta\ (\textsc{Th�venin}\sim\textsc{Norton})
\]
Ampli Op: \textbf{Id�al} $\Rightarrow i^\oplus=i^\ominus=0$ et en r�gime \textbf{lin�aire} $\Rightarrow V^\oplus=V^\ominus$ \\
\[\tau=RC=\frac{L}{R} \hspace{15mm} \mathcal{E}=\frac{1}{2}Li^2=\frac{1}{2}CE^2 \hspace{15mm}\omega_0^2=\frac{1}{L\,C}
\]
\titre{Filtres}
\[ x=\frac{\omega}{\omega_c} \hspace{10mm}
\underline{H}(\jmath \omega)=\frac{\underline{H_0}}{1+\jmath x} \hspace{10mm}
\underline{H}(\jmath \omega)=\frac{\underline{H_0}\:\jmath x}{1+\jmath x} \hspace{10mm}
\]
\[ x=\frac{\omega}{\omega_0} \hspace{10mm}
\underline{H}(\jmath \omega)=\frac{\underline{H_0}}{1-x^2+\jmath \frac{x}{Q}} \hspace{10mm}
\underline{H}(\jmath \omega)=\frac{-x^2\:\underline{H_0}}{1-x^2+\jmath \frac{x}{Q}} \hspace{10mm}
\underline{H}(\jmath \omega)=\frac{\underline{H_0}}{1+\jmath\;Q\left(x-\frac{1}{x}\right)} \hspace{5mm} \Delta x=\frac{1}{Q} 
\]
\indent $\circledast$ R�sonance si $Q>\frac{1}{\sqrt{2}}$. Si $Q\gg 1 \Rightarrow x_r\simeq 1$ et $\beta=Q$ la surtension de r�sonance\\
\titre{Puissance en r�gime sinuso�dal}
\[\mathscr{P}_{instant}=U\,i \hspace{10mm}
\mathscr{P}_{moy}=U_\eff I_\eff \cos \varphi=\Re(\underline{\mathcal{Z}})\; I_\eff^2=\Re(\underline{Y})\; U_\eff^2=
\]
\indent $\circledast$ Adaptation d'imp�dance: $\mathcal{Z}_c=\mathcal{Z}_g^\star$ \\
\titre{Electro \& Magn�to statique}
\[\vec{E}(M)=\frac{1}{4\pi\varepsilon_0}\iiint\frac{\rho(P)}{PM^2}\vec{u}_{PM}\D\tau\ (\textsc{Gauss}) \hspace{20mm}
V(M)=\frac{1}{4\pi\varepsilon_0}\iiint\frac{\rho}{r}\D\tau
\]\[
\vec{B}(M)=\frac{\mu_0}{4\pi}\int \frac{i\,\overrightarrow{\D l}\wedge \vec{u}_{PM}}{r^2}=
\frac{\mu_0}{4\pi}\iiint\frac{\vec{j}\wedge \vec{u}_{PM}}{r^2}\ (\textsc{Biot \& Savart})
\]
\[
\Phi_\mathcal{S}(\vec{E})= \frac{Q_{int}}{\varepsilon_0} \hspace{20mm} \Phi_\mathcal{S}(\vec{g})=-4\pi G M_{int} \hspace{20mm}
\oint_\mathscr{C}{\vec{B}\cdot\overrightarrow{\D l}}=\mu_0 I_{enlac\acute{e}}
\]
\[\D \vec{f}_\textsc{Laplace}=i\,\overrightarrow{\D l}\wedge \vec{B}=\vec{j}\D \tau \wedge \vec{B} \hspace{20mm}
C=\frac{\varepsilon_0 S}{e}
\]
\indent $\circledast$ Dip�les:
\[ \vec{p}=q\,\overrightarrow{NP} \hspace{10mm}
V(r,\theta)=\frac{p\cos\theta}{4\pi\varepsilon_0 r^2} \hspace{10mm}
\vec{E}=-\grad{V}=\frac{1}{4\pi\varepsilon_0 r^3} \left(
\begin{array}{c}
	2p\cos\theta \\
	p\sin\theta
\end{array} \right) \hspace{10mm}
\vec{\Gamma}=\vec{P}\wedge\vec{E}_{ext}
\]\[
\overrightarrow{M}=\vec{S}\,I \hspace{10mm}
\vec{B}=\frac{\mu_0}{4\pi r^3} \left(
\begin{array}{c}
	2M\cos\theta \\
	M\sin\theta
\end{array} \right) \hspace{10mm} \vec{\Gamma}=\overrightarrow{M}\wedge\vec{B} \hspace{10mm}
Ep=-\vec{p}\cdot \vec{E}_{ext}=\overrightarrow{M}\cdot \vec{B}_{ext}
\]
\titre{Loi d'\textsc{Ohm}}
\[I=\diff{Q}{t}=\iint_\mathcal{S}\vec{j}\cdot \overrightarrow{\D S}  \hspace{10mm}
\vec{j}=nq\vec{v} \hspace{10mm} \vec{j}=\gamma\vec{E} \hspace{10mm} \rho=\frac{1}{\gamma}\ (\Omegasl.m)
\]
\titre{�quations de \textsc{Maxwell}}
\[\div{\vec{B}(M;t)} = 0 \hspace{2cm} \rot{E(M;t)} = - \frac{\partial \vec{B}}{\partial t}(M;t) \]
\[\div{\vec{E}(M;t)} = \frac{\rho (M;t)}{\varepsilon_0} \hspace{2cm} \rot{B(M;t)} = \mu_0\,\vec{j}(M;t) + \varepsilon_0 \frac{\partial \vec{E}}{\partial t}(M;t) \]
\titre{Induction}
\[
\vec{E}_m=-\partd{\vec{A}}{t}+\vec{v}_e\wedge\vec{B} \hspace{15mm}
e_{AB}=\int_A^B \vec{E}_m\cdot \overrightarrow{\D{l}} \hspace{15mm}
e=-\frac{\D\Phi}{\D t} \hspace{15mm}
\Phi=L\;i
\]