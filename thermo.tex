\chapitre{Thermodynamique}
\titre{Statique des fluides}
\indent $\circledast$ Relation fondamentale de la statique des fluides et applications: \emph{(Signes � adapter)}
\[\overrightarrow{\D f_p} = -P \overrightarrow{\D S}
\hspace{2cm} \frac{\D P}{\D z}=-\rho\:\mathrm{g}
\hspace{2cm} P(z)=P_0\:\exp\left(\frac{-M_{air}\mathrm{g}\,z}{RT_0}\right) \]
\indent $\circledast$ Pouss�e d'\textsc{Archim�de}: \hspace{10mm} $\vec{\Pi}=\vec{P_A}=-\rho\,V\,\rm{g}$ \\
\titre{Coefficients thermo-�lastiques}
\[\alpha = \frac{1}{V}\left. \partd{V}{T}\right|_P \text{Dilatation thermique} \hspace{1cm} \chi_T= \frac{-1}{V}\left. \partd{V}{P}\right|_T \text{Compressibilit� isotherme}\]
\titre{Equations d'�tat}
\[PV=nRT \hspace{2cm} \left(P+\frac{n^2}{V^2}a\right)\left(V-n\,b\right)=nRT\]
\titre{Capacit�s thermiques}
\[C_V=\left.\partd{U}{T}\right|_V \Leftrightarrow \D U = C_V \D T \hspace{15mm} C_P=\left.\partd{H}{T}\right|_P \Leftrightarrow \D H = C_P \D T \]
\[C_P-C_V=nR \hspace{15mm} \gamma=\frac{C_P}{C_V} \hspace{15mm} C_V=\frac{nR}{\gamma -1} \hspace{15mm} C_P=\frac{nR\gamma}{\gamma -1}\]
\titre{Loi de \textsc{Laplace}}
\[TV^{\gamma -1}=\rm{c}\up{ste} \hspace{15mm} PV^{\gamma}=\rm{c}\up{ste'} \hspace{15mm} T^{\gamma}P^{1-\gamma}=\rm{c}\up{ste''}
\]
\titre{Fonctions d'�tat}
\[ H=U+PV \hspace{15mm} G=H-TS \hspace{15mm} F=U-TS
\]
\[\D U=T\D S-P\D V \hspace{10mm} \D H=T\D S +V \D P \hspace{10mm} \D F=-S \D T -P \D V \hspace{10mm} \D G =-S \D T +V \D P
\]
\titre{Formule de \textsc{Clapeyron}}
\[h_2-h_1=T(v_2-v_1)\left(\frac{\D P}{\D T}\right)_{\acute{e}q}\]
\[\underset{0^\circ C}{\text{Glace}}\xrightarrow{334\:\mathrm{kJ}.\mathrm{kg}^{-1}}\underset{0^\circ C}{\text{Eau}}\xrightarrow[donc\ 418\:\mathrm{kJ}.\mathrm{kg}^{-1}]{4.18\:\mathrm{kJ}.\mathrm{kg}^{-1}.K^{-1}}\underset{100^\circ C}{\text{Eau}}\xrightarrow{2260\:\mathrm{kJ}.\mathrm{kg}^{-1}}\underset{100^\circ C}{\text{Vapeur}}
\]
\titre{Ph�nom�nes de diffusion}
\begin{center}
Loi de \textsc{Stephan}: $\mathcal{P}_{tot} = \sigma\;T^4$ \hspace{15mm} Loi de Wien: $T\;\lambda _{max} \approx 2800\; K.\mu m$ \\
Loi de conservation: $\partd{e}{t}+\div{\vec{\jmath}}=0$\\

Loi de \textsc{Fick}: $\vec{\jmath}_n=-D\;\grad{C}$ o� C en $particules.m^{-3}$ et D en $m^2.s^{-1}$\\
Loi de \textsc{Fourier}: $\vec{\jmath}_Q=-\lambda\;\grad{T}$ o� $\vec{\jmath}_Q$ en $W.m^{-2}$
\end{center}

